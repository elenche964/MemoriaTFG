  \section{Motivación}\label{sec.motivacion}
  
  %Estudiar las problemáticas típicas de las personas con Alzheimer

    La demencia se puede definir como el deterioro adquirido de las capacidades cognitivas que entorpece la realización satisfactoria de las actividades diarias. El principal causante de demencia, que se estima entre un 60 y un 70 por ciento, es la Enfermedad de Alzheimer (EA), una enfermedad degenerativa que se da habitualmente en pacientes mayores de 65 años. Es importante en este tipo de enfermedades considerar la poca información que se tiene sobre la enfermedad y que, hasta ahora, no tiene cura. Esto se traduce en que no sólo afecta al sujeto, sino también a sus cuidadores y familiares por el grado de dependencia y discapacidad que se puede llegar a sufrir durante la EA.\cite{BarreraLopez2018}\\
 
    %	Cambiar lo que sigue, no me gusta 
    %	Es un campo bastante inexplorado dada su complejidad (es un campo desconocido porque es bastante incierto)

    %Esta enfermedad, caracterizada por el cúmulo de ciertas proteínas en el cerebro, condiciona un deterioro cognitivo progresivo e invariable, afectando principalmente a la memoria reciente. 
 En este trabajo, nos centramos en los pacientes que padecen la EA en una fase temprana y su entorno. En esta fase se observa, aparte de la pérdida de memoria reciente, episodios de ansiedad y falta de asociatividad, concentración, sueño y apetito, entre otros. Esto supone tanto para la persona que padece EA como para su entorno un aumento de responsabilidades en el cuidado, dándose una gran pérdida de autonomía en todas las partes implicadas y llegando a generar problemas graves de ansiedad y depresión.\\
 
 
 %importante poner en la motivación lo que puecde ocurrir con la pérdida de memoria, como la deorientación y pérdida de unapersona en la calle
  %La siguiente frase la quiero cambiar
   Una plataforma de servicios orientada y creada especialmente atendiendo a las necesidades de todas las personas implicadas, puede ofrecer no sólo tranquilidad al llevar un seguimiento, monitorización o información relevante sobre la persona al cuidado, sino también una mayor autonomía. Todo esto se traduce en un aumento de la calidad de vida, en concreto, colocar un pequeño dispositivo de localización sobre la persona que padece EA, disminuye la dependencia del sujeto. Utilizando IoT junto a elementos como una etiqueta NFC, puede ayudar a llevar a cabo tareas cotidianas, como olvidar qué comprar, que se empiezan a ver afectadas desde el inicio de la EA.\\

 