 \section{Recursos utilizados}\label{sec.recursos}
 %Se irá rellenando conforme vayamos utilizando diferente tecnología (describir con un par de líneas)
 %poner en ionic que se utiliza Angular y Node.js
 
 %El sistema operativo es Microsoft Windows 10 Pro Versión 10.0.19041 compilación 19041
 
 \subsection*{Recursos software}
 
 \begin{enumerate}[wide=\parindent, label= ]
 	\item Angular: Framework modular y escalable diseñado para el desarrollo de aplicaciones web utilizando el patrón MVC (Modelo-Vista-Controlador). Su lenguaje de programación es TypeScript, una ampliación de JavaScript. Versión: 8.3.25
 	\item AdobeXD: Herramienta de edición gráfica orientada a la creación de interfaces tanto web como móviles. Se utiliza para la creación de Mockups. Versión: 36.2.32.5
 	\item Node.js: Entorno de ejecución de JavaScript diseñado para la creación de aplicaciones web escalables y orientado a eventos asíncronos. En este proyecto lo utilizaremos tanto como motor JavaScript  para Angular, como para crear servicios REST. Versión: 14.15.4
 	\item MySQL Server: Sistema de gestión de bases de datos relacionales usando el modelo cliente-servidor. Versión: 8.0.23
 	\item MySQL Workbench: IDE para trabajar con el sistema SQL. Versión: 8.0
 	\item JetBrains Product Pack for Students: Paquete gratuito de IDEs ofrecido a estudiantes de mano de JetBrains. Haremos uso de PyCharm y WebStorm. Versiones: 2020.3
 	\item Django: Framework de alto nivel para el desarrollo web utilizando el patrón MVC (modelo-Vista-Controlador). Desarrollado en Python, lo utilizaremos para crear servicios. Versión: 3.1.7
 	\item Python: Lenguaje de programación interpretado. Versión: 3.8.6
 	\item JavaScript: Lenguaje de programación para web.
 	\item MongoDB: Sistema de gestión de bases de datos NoSQL usando el modelo cliente-servidor orientado a documentos de código abierto. Versión: 4.4.4
 	%\item MongoDB Compass: al final no lo he usado na mas que pa bichear más cómodo que con mongo
 	\item Ionic: Framework para el desarrollo de aplicaciones multiplataforma. Como framework de interfaz utilizaremos Angular con motor Node.js. Versión: 6.12.4
 	\item Postman: Herramienta para hacer tests de APIs REST. Versión: 8.1.0
 	\item Mendeley Desktop: Gestor de referencias bibliográficas multiplataforma. Versión 1.19.4
 	\item \LaTeX\ : Sistema de composición de textos. El editor utilizado es TeXworks y la distribución MikTeX.
 	\item git: Sistema de control de versiones distribuido. Los repositorios se localizan en GitHub. Versión: 2.28.0.windows.1
 \end{enumerate}
 
 \subsection*{Recursos hardware}
 
 \begin{enumerate}[label=]
 	\item 
 \end{enumerate}