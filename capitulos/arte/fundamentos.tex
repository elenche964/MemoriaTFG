  \section{Fundamentos}\label{sec.fundamentos}
  
  Desde hace más de una década se utiliza el término eHealth consistente en, según la OMS desde 2005,  \enquote{el apoyo que la utilización costoeficaz y segura de las tecnologías de la información y las comunicaciones ofrece a la salud y a los ámbitos relacionados con ella,  con  inclusión  de  los  servicios  de  atención  de  salud,  la  vigilancia  y  la  documentación  sanitarias,  así como la educación, los conocimientos y las investigaciones en materia de salud}\cite{OrganizacionMundialdelaSalud2005}. Este término encaja perfectamente con nuestro proyecto, una plataforma de servicios ofrecido especialmente para atender la salud de los pacientes objetivos.\\
  
  \par
  El avance en las tecnologías aplicadas a la eHealth nos lleva a otro término muy extendido y que engloba gran parte de los elementos que forman parte de nuestro día a día, el \enquote{IoT}, del inglés, \enquote{\textit{Internet of Things}}. Este término hace referencia a la interconexión existente entre todo tipo de dispositivos, desde un sensor de movimiento o una cámara de vigilancia hasta cualquier tipo de dispositivo móvil complejo o incluso industrial, como puede ser un Smartphone.\\
  
  \par
  Para realizar la conexión entre estos dispositivos que forman parte de una red entre sí y con la información con la que trabajan, están cada vez más extendidos los denominados \enquote{\textit{microservicios}}, que dividen el software en elementos muchos más pequeños, lo que aporta una gran escalabilidad, la posibilidad de trabjar en distintos lenguajes según el servicio que se quiera ofrecer o una mayor capacidad de recuperación, ya que los microservicios pueden trabajar de forma independiente; en contraposición, añade complejidad al diseño, ya que tienen que estar coordinados entre sí y las pruebas pueden llegar a ser algo tediosas.\\
  
  \par
  En cuanto a la tecnología actual sobre computación y almacenamiento de datos, podemos decir que el futuro es lo que conocemos como \enquote{Cloud Computing}. Actualmente, se ofrecen todo tipo de servicios en la nube a nivel computacional; lo más importante es que únicamente se paga por los recursos utilizados y te permite de manera dinámica aumentar o disminuir los servicios contratados en la mayoría de los casos.\\
  
  \par
  
  Por otro lado, el desarrollo software se lleva a cabo actualmente a través de \enquote{Frameworks}, tanto para el desarrollo de aplicaciones como de servicios. Los más utilizados actualmente son \enquote{Ionic} y \enquote{React Native} que permiten la creación de aplicaciones multiplataforma. En cuanto a la prueba de sistemas, cada vez está más extendida la técnica de Sandwich, que supone un híbrido entre un test del sistema completo a cada componente (de arriba hacia abajo) y de cada componente hasta el sistema completo. Esta técnica persigue probar en paralelo la interfaz de usuario y los componentes software del sistema para en una etapa final, hacer pruebas sobre el sistema completo; esto permite ganar tiempo en pruebas, pero necesita bastantes drivers para las pruebas y puede tener un coste bastante elevado.\\
  
  
  
  