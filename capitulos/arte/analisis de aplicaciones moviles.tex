  \section{Análisis de aplicaciones móviles}\label{sec.analisis de aplicaciones moviles}
  %En cada subsección, explicar quién lo propone y por qué lo propone. Cuándo lo propone, sus características. Ventajas e inconvenientes.
  
  \subsection*{Alzheimer Master}
  
  	Esta aplicación, creada por un empresa Húngara y publicada en 2017 está disponible para Android. Su función es crear notificaciones con mensajes de voz o video personalizados y grabas la reacción de la persona afecta ante el mensaje. Su objetivo es combatir la soledad y ofrecer apoyo a la persona afectada por la enfermedad.\\
  	
  	\par
  	
  	Existe una versión gratuita, que permite crear una notificación y una grabación con el objetivo de probar si esta se ajusta a las necesidades del usuario. En su versión de pago se eliminan las restricciones y permite conectart la aplicación con un \enquote{Android Smart Watch}, añadiendo la posibilidad de realizar acciones cuando el usuario se despierta de noche y así evitar la desorientación. Como mejora de futuro, se está trabajando en añadir una función para saber cuándo sale de casa.\\
  	\par
  	
  	\textbf{Ventajas:}
 \begin{enumerate}[label=-]
	\item Notificaciones totalmente personalizadas.
	\item Se puede grabar cómo reacciona a los mensajes.
	\item Permite bloquear la aplicación con un pin para que el afectado no pueda configurarla.
\end{enumerate}

	\textbf{Inconvenientes:}
\begin{enumerate}[label=-]
	\item Hay que configurar la aplicación y ver las reacciones desde el dispositivo de la persona que la va a utilizar.
	\item Para funcionalidades más complejas se necesita de un dispositivo Smart Watch.
	\item No se puede personalizar la aplicación.
	\item En caso de olvidar el pin no se puede acceder a la aplicación.
\end{enumerate}

\subsection*{Timeless}

Esta aplicación, creada por Emma Yang en 2018 con tan sólo 14 años, nació con el objetivo dede ayudar a su abuela. Ofrece una conexión en tiempo real entre paciente, cuidador y familia/amigos. Permite enviar fotos con etiquetas de identificación, programar eventos y, con ayuda del localizador del dispositivo, saber el tiempo, una galería de fotos de personas, la opción de identificar automáticamente a la persona a la que se le hace la fotografía desde la app y realizar llamadas a las personas que así se haya configurado.\\

\par

Desde su creación, la aplicación se ha seguido desarrollando y ya se ofrece en varios idiomas para Mac, iPhone y iPad. La financiación para este proyecto que aún sigue en marcha con planes de mejora como añadir gamificación o el uso de inteligencia artificial para detectar hábitos o anomalías en el comportamiento, vino a través de la plataforma \enquote{Indiegogo}, con un presupuesto inicial de $8.837€$.\\

\par

  	\textbf{Ventajas:}
 \begin{enumerate}[label=-]
	\item El cuidador es el que crea y maneja los eventos del paciente.
	\item Las aplicaciones están sincronizadas.
	\item Permite estar en total contacto con el entorno en una única aplicación.
	\item Utiliza Inteligencia Artificial para el reconocimiento de personas dentro del entorno a partir de una fotografía.
	\item Se accede a la palicación a través de la huella digital o reconocimiento facial.
\end{enumerate}

	\textbf{Inconvenientes:}
\begin{enumerate}[label=-]
	\item Únicamente está disponible para dispositivos Apple.
	\item Requiere un aprendizaje por parte del paciente para el manejo.
	\item No permite utilizar la ubicación para saber dónde se encuentra el paciente.
\end{enumerate}
