\section{Análisis de sistemas existentes}\label{sec.analisis de sistemas existentes}
   %En cada subsección, explicar quién lo propone y por qué lo propone. Cuándo lo propone, sus características. Ventajas e inconvenientes. Hablar de sistemas informáticos, pero tb de sistemas asistenciales.
   
  \subsection*{Jiobit}
  
  	Jiobit es un sistema de seguimiento en tiempo real que fue diseñado por Jhon Renaldi, que tras perder a su hijo en un parque en 2015, buscó dispositivos para localizar en tiempo real, pero no encontró ningún dispositivo \enquote{wearable} adecuado. Buscaba que fuera elegante, tuviera una batería duradera y no dependiera de un dispositivo, ya fuera por necesitar de conexión bluetooth o de cualquier otro tipo. \\
  	  	\par
  	\begin{wrapfigure}{l}{0.25\textwidth}
  		\centering
		\includegraphics[width=0.25\textwidth]{imagenes/Jiobit.jpg}
	\end{wrapfigure}
  	

  	Aunque este dispositivo fue diseñado para su uso en niños, por su versatilidad pronto se extendió a mascotas, adultos y personas mayores. Tiene el tamaño de una galleta y dispone de un botón de alerta para avisar en tiempo real, el dispositivo se conecta a un servicio en la nube que a su vez conecta con los dispositivos a los que esté asociado Jiobit. \\
  	\par
  	
  	\textbf{Ventajas:}
  	
\begin{enumerate}[label=-]
  		\item No necesita de otro dispositivo para enviar la posición actual al servidor.
  		\item Es ligero y fácil de colocar.
  		\item Cuenta con un botón de aviso.
  		\item Permite establecer sitios seguros.
\end{enumerate}

  	\textbf{Inconvenientes:}
  	
\begin{enumerate}[label=-]
  		\item Hay que pagar mensualmente por el servicio.
  		\item Es fácil que el paciente se deshaga del dispositivo al ser un \enquote{wearable} bastante visible.
  		\item Puede que se olvide la recarga del dispositivo.
\end{enumerate}

\subsection*{Alexa Eldercare Toolbox\cite{Tan2020}}

	Este sistema fue propuesto por investigadores del \enquote{Instituto Tecnológico de Stevens} en 2020, partiendo de estudios en los que la población anciana prefiere quedarse en su hogar. tomando como base el coste anual que puede suponer el cuidado de una persona mayor, se diseñó un sistema utilizando un dispositivo Amazon Echo Dot. De esta manera, también se evita el cuidado en casa, reduciendo el riesgo de contagio de COVID-19.\\
	
	\par
	La potencia de este sistema es la amplia gama de necesidades que puede cubrir, algunas de las que se cubrieron con este sistema fueron: ayudar en la realización de tareas diarias a través de instrucciones, control del tiempo en la cocina, ayuda con la dieta o riesgos de caídas. \\
	
	\par
	  	\textbf{Ventajas:}
  	
\begin{enumerate}[label=-]
  		\item Cubre muchísimas necesidades dentro del hogar.
  		\item Se puede configurar facilmente a través de una aplicación.
  		\item Totalmente personalizable.
  		\item Contacto directo en tiempo real.
\end{enumerate}

  	\textbf{Inconvenientes:}
  	
\begin{enumerate}[label=-]
  		\item No avisa cuando sale de casa.
  		\item Depende de una conexión a internet.
  		\item Puede que se olvide la recarga del dispositivo.
  		\item Amazon Echo se debe colocar en un sitio estratégico de cada domicilio para que esté en continuo contacto.
  		\item El paciente puede rehusarlo, puesto que es un cambio significativo para la vida cotidiana.
\end{enumerate}

%No sé de qué más tipos de sistemas hablar, hay de todo.